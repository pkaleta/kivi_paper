\documentclass[12pt,a4paper]{report}
%\documentclass[a4paper]{report}
\usepackage{fullpage}
\usepackage{alltt}
\usepackage{listings}
\usepackage{caption}
\usepackage{url}
\usepackage{amsthm}
\usepackage{setspace}
\newtheorem{definition}{Definition}[chapter]
%\usepackage{courier}

\begin{document}

\lstset{
  basicstyle=\small\sffamily,           % the size of the fonts that are used for the code
%  frame=lines,
  numberstyle=\footnotesize,      % the size of the fonts that are used for the line-numbers
  stepnumber=2,                   % the step between two line-numbers. If it's 1, each line 
  % will be numbered
  numbersep=5pt,                  % how far the line-numbers are from the code
  showspaces=false,               % show spaces adding particular underscores
  showstringspaces=false,         % underline spaces within strings
  showtabs=false,                 % show tabs within strings adding particular underscores
  tabsize=4,                      % sets default tabsize to 2 spaces
  captionpos=b,                   % sets the caption-position to bottom
  breaklines=true,                % sets automatic line breaking
  xleftmargin=20pt,
  xrightmargin=20pt,
  breakatwhitespace=false,        % sets if automatic breaks should only happen at whitespace
  title=\lstname,                 % show the filename of files included with \lstinputlisting;
  escapeinside={\%*}{*)},         % if you want to add a comment within your code
  morekeywords={*,...}            % if you want to add more keywords to the set
}

\lstdefinelanguage{LLVM}
  {morekeywords={define, ret, call, load, add, sub, store, switch, bitcast,
  ptrtoint, inttoptr, label, global, undef},
  sensitive=false,
  morecomment=[l]{;},
  morestring=[b]",
}

\lstdefinelanguage{Haskell}
  {morekeywords={data, type, where, let, Just, Nothing, deriving, Show, letrec,
  in},
  sensitive=true,
  morecomment=[l]{;},
  morestring=[b]",
}


\lstdefinestyle{assembler}{
  language=LLVM
}
\lstdefinestyle{haskell}{
  language=Haskell
}

\title{Kivi: Lazy functional programming language targeting Low Level Virtual Machine}
\author{\textit{Author:}\\Piotr Micha\l{} Kaleta\\\\\emph{Supervisor:}\\dr W\l{}odzimierz Moczurad}
\date{\today}

\maketitle
\pagenumbering{roman}

\newpage
\thispagestyle{empty}
\mbox{}

\Huge
\begin{abstract}
  \normalsize
  \center
  This paper describes the design and Haskell implementation of a lazy
  programming language called Kivi. It is higly influenced on both Haskell's
  syntax as well as the way that computations are performed. However it generates
  the \textit{Low Level Virtual Machine}\cite{website:llvm} Intermediate
  Representation code as the output, which is then assembled and compiled to
  native code by LLVM.
\end{abstract}


\renewcommand{\abstractname}{Acknowledgements}
\begin{abstract}
  \normalsize
  \center

  TODO: napisac

  \begin{flushright}
    Piotr Kaleta
  \end{flushright}
\end{abstract}

\normalsize

\tableofcontents

\onehalfspace

\pagenumbering{arabic}
\chapter{Introduction}

This paper describes the design and implementation of a lazy functional
language compiler. The implementation is based on the
\textit{G-machine}\cite{Jon87} and uses \textit{lazy graph reduction} to
perform evaluation.

The compilation process is implemented as a set of passes that eventually
transforms the source code into simple intermediate language \textit{lambda
calculus} enriched with let(rec) expressions. It would be possible to implement
compiler using less passes than I did, or maybe even one, but such clear
separation made things simpler to both develop and debug as well. Besides such
separation, made the compiler very modular and inserting additional passes
could be done with no minimal complexity.


In following chapters I am going to describe the process of translating
\textit{Lambda Calculus} into \textit{G-code} instructions.  Finally these
instructions will either be evaluated by the interpreter or compiled down to
LLVM intermediate representation and further to native code by LLVM
infrastructure compilers. Design and implementation of both of these concepts
will be described in separate chapters.

TODO: Opis po kolei czesci dokumentu

\section{Implementation language - Haskell}


\chapter{Parsing}

In order to translate the source of \textit{Kivi} into \textit{Lambda Calculus}
it has to be available in a structured form so thate later passes could easily
traverse it. \textit{Abstract Syntax Trees}\cite{wiki:ast} has been widely adopted as a
standard form of keeping the representation of source code in memory. The
process of reading the input source file and producing AST from it is called
\textit{parsing}.
I've chosen to implement a simple parser, in a similar manner as in
\cite{JonLes00} as it seemed as the simplest approach to me then. Now, as
the parser has grown, I'd strongly consider rewriting it to use parser combinator
library such as \cite{website:parsec} instead.  The parsing consists of two
passes. As previously mentioned such separation makes implementation and
reading the code easier. Thus parsing consist of:

\begin{itemize}
  \item Firstly \textit{lexical analysis} implemented by \texttt{lex} in
    \textit{Lexer.hs} module extracts \textit{tokens} from plain text.
    \textit{Tokenization} is the process of demarcating and possibly
    classifying sections of a string of input characters. The resulting tokens
    are then passed on to further processing.
  \item \textit{Syntax analysis} is the process of analyzing the stream of
    tokens and building an \textit{Abstract Syntax Tree} from it.
\end{itemize}


TODO: Diagram of how parsing process work

TODO: Grammar of the language

This two-phase process is implemented as\footnote{Where \texttt{PatProgram}
is program containing patterns}:

\hspace*{-1.5in}
\begin{lstlisting}
parse :: String -> PatProgram
parse = syntax . lex
\end{lstlisting}

\section{Lexical analysis}

The types for lexer are following:

\hspace*{-1.5in}
\begin{lstlisting}
type Token = String
type TokenInfo = (Int, Token)
\end{lstlisting}
Where \texttt{TokenInfo} consist of line number and a token string itself.

The heart of \textit{lexical analyser} is the \texttt{lex'} function. It works
by classifying the parts of remaining input based on first characters.
Building number tokens may be a nice example of how it works:

\hspace*{-1.5in}
\begin{lstlisting}[label=lst:lex_comment,caption={Building tokens from numbers.}]
lex' (c : cs) lnum | isDigit c =
    (lnum, numToken) : lex' restCs lnum
    where
        numToken = c : takeWhile isDigit cs
        restCs = dropWhile isDigit cs
\end{lstlisting}


\section{Syntax analysis}
\label{sec:syntax_analysis}
Once we can accept tokens from lexer it's time to build a hierarchical
structure called \textit{Abstract Syntax Tree} (or AST for short) from it. ASTs
are tree representations of abstract syntactic structures written in
programming language. Nodes in such tree represent different constructs from
the source code. The word abstract is used because not every detail from source
code has its counterpart in the AST, some unnecessary details are ommited.

Data type for representing Kivi's Abstract Syntax Trees is defined in
\textit{Common.hs} module and it is a parametrized data type. This concept is
a similar idea to templates in C++ or generics in Java. It allows the code to
be reused and forms the basic idea of \textit{parametric poylmorphism}
(see \cite{website:parametric_polymorphism}). The data type definition is
following:

\hspace*{-1.5in}
\begin{lstlisting}
data Expr a = EVar Name
            | ENum Int
            | EChar Int
            | EConstrName Name
            | EConstr Int Int
            | EAp (Expr a) (Expr a)
            | ELet IsRec [Defn a] (Expr a)
            | ECase (Expr a) [Alter Pattern a]
            | ECaseSimple (Expr a) [Alter Int a]
            | ECaseConstr (Expr a) [Alter Int a]
            | ELam [a] (Expr a)
            | EError String
            | ESelect Int Int Name
\end{lstlisting}
The \texttt{Expr} data type is parametrized with respect to its binders.
Binders, as name suggests, are used when the variable is being bound to a
value. This happens in lambda abstractions, \texttt{let} and \texttt{letrec}
definitions as well as in \texttt{case} expressions. To give an e example of
how a simple source code construct will look, when parsed to AST,  lets look at
the following example:

\hspace*{-1.5in}
\begin{lstlisting}
let
    x = 1;
    y = 2
in
    x + y
\end{lstlisting}
The result returned from \texttt{parse} function will be\footnote{In reality
the output from \texttt{parse} is more complicated because it returns programs
with patterns. However in order to grasp the concept of parametrized
\texttt{Expr} type its easier to forget about that fact for a while.}:

\hspace*{-1.5in}
\begin{lstlisting}
  ELet False
       [('x', ENum 1), ('y', ENum 2)]
       EAp (EAp (EVar '+') (EVar 'x')) (EVar 'y')
\end{lstlisting}

Defining expressions in this way greatly reduces the amount of code to be
written. For example, initially parser returns programs with patterns in them,
but later phases of compilation process gets rid of them by transforming
program into a representation that uses \texttt{case} expressions instead.
Thanks to parametrized definition of \texttt{Expr} data type we could define
parse as returning \texttt{Expr Pattern}, whereas later phases return
\texttt{Expr String}.

\section{Implementation of Parser}

TODO: tu mozna napisac jak dziala parser o ile za duzo nie wyjdzie reszty

\subsection{Analysing patterns}

TODO: co tu mialo byc?


\chapter{Translation to Lambda Calculus}

In this chapter I'm going to present the process of transformation of a
high-level functional language into a simple intermediate form - \textit{lambda
calculus}. In the first part I'll show \textit{Kivi}'s syntax, its
constructs and structures. Later on the desired output form -
\textit{Lambda Calculus} will be shown. The rest of the chapter will
consist of ways to translate the first form into another.

\section{Syntax}
Source of \textit{Kivi}'s program consist of set of
\textit{supercombinators}\cite{wiki:supercombinator}. Supercombinator is an
expression consisting of either constants or other supercombinators and doesn't
contain any free variables (see Definition~\ref{def:free_variable}) in its
body. Supercombinators might have arguments, and those which doesn't are called
\textit{constant applicative forms} (or CAFs for short).
Listing~\ref{supercombinator_ex} shows the simple supercombinator with one
argument.

\hspace*{-1.5in}
\begin{lstlisting}[style=haskell,label=supercombinator_ex,caption={Simple supercombinator.}]
sqr x = x * x
\end{lstlisting}

There is one special supercombinator called \textit{main}, that takes no
arguments. It is the entry point for program execution. If there's no
\textit{main} CAF present in Kivi's source, both compiler and interpreter
should issue specific error message.

\subsection{Case expressions}
The purpose of case expression is to allow the programmer to control the flow
of program execution via a multiway branch. The semantics of case is rather
simple. First, the expression under case is evaluated and then based on result,
the appriopriate execution path is chosen. Example of case expression is
presented in Listing~\ref{lst:case_ex}.

\hspace*{-1.5in}
\begin{lstlisting}[style=haskell,label=lst:case_ex,caption={Fibonacci with case}]
fib n =
    case n of
        0 -> 1;
        1 -> 1;
        n -> fib (n-1) + fib (n-2);

main = fib 10
\end{lstlisting}

\subsection{If expressions}

TODO: write

TODO: zmienic \ref{xxx} na \ref{lst:xxx}

\subsection{Local bindings}
Supercombinators provides the ability to define local definitions using
\texttt{let} keyword. The scope of variables defined using \texttt{let} is
enclosed by the expression following \texttt{in} keyword. The example use of
local binding is presented in Listing~\ref{let_ex}

\hspace*{-1.5in}
\begin{lstlisting}[style=haskell,label=let_ex,caption={Local \texttt{let} binding.}]
sum x y = x + y;

main =
    let x = 1
        y = 2
    in
        sum x y
\end{lstlisting}

\subsection{Recursive local bindings}
In order to define mutually recursive bindings using \texttt{let} one has to use the
special \texttt{letrec} construct as shown in Listing~\ref{factorial_letrec_ex}

\hspace*{-1.5in}
\begin{lstlisting}[style=haskell,label=factorial_letrec_ex,caption={Factorial function using \texttt{letrec}.}]
sum x y = x + y;

main =
    letrec fac n =
        case n == 0 of
            True  -> 1;
            False -> n * fac (n-1)
    in
        fac 5
\end{lstlisting}

\subsection{Where clauses}
Another way of creating recursive definitions is using the \texttt{where}
clause. Internally it will be translated into letrec binding. Example use of
\texttt{where} clause is shown in Listing~\ref{factorial_where_ex}

\hspace*{-1.5in}
\begin{lstlisting}[style=haskell,label=factorial_where_ex,caption={Factorial function using \texttt{where}.}]
main = fac 5
    where
        fac n = case n == 0 of
            True  -> 1;
            False -> n * fac (n - 1)
\end{lstlisting}

\subsection{Lambda abstractions}
Functions in Kivi can be defined in two ways. The first one is the top level
supercombinators that has already been discussed. The other option is to define
them as \textit{lambda abstractions}. This concept is similar to \textit{anonymous
functions} used in imperative languages. The syntax for defining lambda
abstractions is presented in Listing~\ref{lambda_ex}.

\hspace*{-1.5in}
\begin{lstlisting}[style=haskell,label=lambda_ex,caption={Lambda abstraction}]
f = (\x . x * 2) 42
\end{lstlisting}

In this example the lambda abstraction which doubles the arguments value is
created and then applied to 42 yielding 84 as result.

During compilation process programs containing lambda abstractions are
transformed into their equivalents with lambda abstractions substituted for top
level supercombinators. This process is called \textit{lambda lifting} and is
described in more detail in chapter~\ref{lambda_lifting}.

\subsection{Pattern matching}

Pattern matching consists of specifying patterns to which some data should
conform and then checking to see if it does, as well as deconstructing the data
according to those patterns. So in other words, using pattern matching you can
recognize values, bind variables to those values and break structures down into
parts.
Patterns are matched in order they are defined in source code. Once a
successful branch is found, the right-hand-side expression is evaluated and
result returned. None of the following patterns is checked. If after checking
all patterns it turns out that none of them matches the argument, the error is
returned.

\hspace*{-1.5in}
\begin{lstlisting}[style=haskell,label=pattern_matching_ex,caption={Factorial using pattern matching.}]
fac 0 = 1
fac n = n * fac (n - 1)
\end{lstlisting}

In Listing~\ref{pattern_matching_ex} a recursive supercombinator \texttt{fac}
calculating factorial is defined. There are two cases\footnote{To be honest
there are three cases, but for simplicity reasons we assume that
\texttt{fac} is called only for non-negative integers}. Either the
argument is 0 and then the result is 1, or argument is other than than 0 and
then we progress recursively. The factorial definition is expressed very
clearly by means of pattern matching.
Pattern matching is also very useful when it comes to dealing with data
types as we'll see in next section.

\subsection{Structured Data Types}
Data types in Kivi are defined using \texttt{data} keyword, giving the name of
new type as well as its constructors and arities. Together with pattern matching they
provide a very powerful mechanism for dealing with structured entities.

\hspace*{-1.5in}
\begin{lstlisting}[style=haskell,label=data_type_ex,caption={Calculating length of list.}]
data List = Nil 0 | Cons 2;

length Nil = 0;
length (Cons x xs) = 1 + length xs
\end{lstlisting}

In Listing~\ref{data_type_ex} a data type \texttt{List} is declared as well as
supercombinator calculating the length of a list by means of pattern matching.
In first case argument is pattern matched to the \texttt{Nil} constructor. If
matching succeeds it means that the list is empty, therefore its length is 0.
In next case the argument is matched to second \texttt{List} constructor, that
is \texttt{Cons}. If argument matches, the \textit{tail} of the list is bound to
\texttt{xs} variable and right hand side of that branch is evaluated.

\subsection{Function application and Currying}

TODO: write

\section{Lambda Calculus}

\section{Translation to Lambda Calculus}
The process of translating a high-level functional language to \textit{lambda
calculus} might be considered as a set of program transformations. Each such
transformation is meant to simplify \footnote{or help with further
transformations} the program given as its input, eventually leading to the
expected \textit{enriched lambda calculus} form. All these transformations
accept one form of program as input and yield the transformed program as
output. We might, therefore, regard the compilation process as a function
composition of all transformations. This is exactly the way the compilation is
implemented in Kivi, and Haskell provides means to express this in a very
concise way:

\hspace*{-1.5in}
\begin{lstlisting}[style=haskell]
run :: String -> String
run = showResults
    . eval
    . compile
    . lambdaLift
    . lazyLambdaLift
    . analyseDeps
    . transformToLambdaCalculus
    . mergePatterns
    . tag
    . parse
\end{lstlisting}

We have already discussed parsing in previous chapters, so in the following
sections I'm going to provide a description of all remaining phases.

\section{Structured Data Types and Tagging}
In order to understand the need for this pass one has to know how structured
data types and their constructors are represented internally.
A custom data type declaration in Kivi might look like this one:

\hspace*{-1.5in}
\begin{lstlisting}[style=haskell]
data Tree = Leaf 1 | Branch 2
\end{lstlisting}

This statement declares the new data type called \texttt{Tree} which can be
constructed in two ways. The common way of looking at the constructors is to
consider them as functions with zero or more arguments that return an instance
of a data type as a result. The \texttt{Tree} constructor is \texttt{Leaf} and
it has no further descendants. It takes one argument which is some kind of
value. The other way to construct a \texttt{Tree} is to create a \texttt{Branch}
taking two children (which also are \texttt{Tree}s) as arguments. For example,
conceptual representation of the following tree is shown in Figure~\ref{fig:tree}:

\hspace*{-1.5in}
\begin{lstlisting}[style=haskell]
Branch (Leaf 1) (Branch (Leaf 2) (Leaf 3))
\end{lstlisting}

\begin{figure}[h!]
  \centering

  TODO: przyklad

  \caption{Example of a tree.}
  \label{fig:tree}
\end{figure}

Constructors are used not only for object creation but also for decomposition
into parts as well as distinguishing the different types of trees based on
constructor. This feature is heavily used when pattern matching. For example
the code in Listing~\ref{lst:sqr_tree} uses it to create a new tree with values
that are square roots of values in original tree.

\hspace*{-1.5in}
\begin{lstlisting}[label=lst:sqr_tree,caption={Creating a `square rooted` tree.}]
sqrTree (Leaf v) = Leaf (v * v)
sqrTree (Branch t1 t2) = Branch (sqrTree t1) (sqrTree t2)
\end{lstlisting}

\subsection{Built-in Data Types}
There are few types that are considered built-ins. These are lists, tuples,
and booleans. The ability to think of these common concepts in functional
programming as Structured Data Types makes the implementation straightforward
as I didn't have to treat these entities separately, instead I used a common
notion of data types.
\subsubsection{Booleans}
Boolean values in Kivi are implemented as a regular Structured Data Type having
two constructors representing true and false:

\hspace*{-1.5in}
\begin{lstlisting}[style=haskell]
data Boolean = True | False
\end{lstlisting}

\subsubsection{Lists}
Lists are structures that allow storing a number of items. They
provide a set of operations on them that allowing to create other
lists\footnote{In Kivi lists as well as other types are immutable, which means
that you cannot modify its contents. Instead you create new values which may
differ from previous ones. }
Kivi provides a variety of way to define a list. First of all lists can be seen
as regular Data Type. Thus there are two constructors for a \texttt{List}
data type. \texttt{Nil} and \texttt{Cons}:

\hspace*{-1.5in}
\begin{lstlisting}[style=haskell]
data List = Nil 0 | Cons 2
\end{lstlisting}

\texttt{Nil} instantiates an empty \texttt{List} whereas \texttt{Cons} is meant
to \textit{construct} \texttt{List} given two elements: the \textit{head} and
\textit{tail}. List are recursive structures, thus \textit{tail} is another
list.

It turns out that creating a \texttt{List} is a task that occurs very often in
functional languages, therefore it would be nice not to have to write long
constructs such as this one to create 4-element \texttt{List}:

\hspace*{-1.5in}
\begin{lstlisting}[style=haskell]
Cons 1 (Cons 2 (Cons 3 (Cons 4 Nil)))
\end{lstlisting}

Most functional languages allow programmer to express it in a much shorter way,
and Kivi does it too:

\hspace*{-1.5in}
\begin{lstlisting}[style=haskell]
[1, 2, 3, 4]
\end{lstlisting}

The same thing concern constructing lists given two elements. Kivi provides
`:`(colon) as construction operator for \texttt{List} type, so the programmer
would write:

\hspace*{-1.5in}
\begin{lstlisting}[style=haskell]
((2+3) : [1,2,3,4])
\end{lstlisting}
instead of:

\hspace*{-1.5in}
\begin{lstlisting}[style=haskell]
Cons (2+3)  [1,2,3,4]
\end{lstlisting}

\subsubsection{Tuples}
Tuples are yet another way of storing multiple values in a single one. However
the difference between them and lists is significant. Tuples are
\textit{immutable} in a way that you cannot \textit{cons} to a tuple nor
decompose them into parts using pattern matching. Their main usage is when a
programmer knows in advance how many elements will he need to store.

There is a predefined number of tuple types that you can create, and currently
the bound is set to 4. This means that programmer is not allowed to create
tuples that contain more than 4 elements. This design decision is due to the
fact that from my perspective it's better to use custom Structured Data Type
for readability reasons. The limit of 4 however, is an experimental value and
might be changed.

This all means that tuples with different number of elements are different
types. Therefore there is no such thing as generic `Tuple` type. There are
however \texttt{Tuple0}, \texttt{Tuple1}, \ldots etc., representing tuples with
0 elements, 1 element, \ldots etc.

There are two ways to create a tuple either by using constructors or by a
special \textit{syntactic sugar} construct, similarly as in the case for lists.
These two ways for instantiating a 3-element tuple are equivalent, with the
preference for the second one, due to readability concerns:

\hspace*{-1.5in}
\begin{lstlisting}[style=haskell]
(((Tuple3 1) 2) 3)
\end{lstlisting}
and

\hspace*{-1.5in}
\begin{lstlisting}[style=haskell]
(1, 2, 3)
\end{lstlisting}

\subsection{Implementation and tagging}
Internally each data type consist of a list of constructors, each containing a
\textit{tag}, and \textit{arity}. Constructor tags should uniquely identify
each constructor therefore each tag should be different. The definitions
for \texttt{DataType} and \texttt{Constructor} type synonyms, as in
\textit{Common.hs} are as follows:

\hspace*{-1.5in}
\begin{lstlisting}[style=haskell]
type DataType = (Name, [Constructor])
type Constructor = (Name, Int, Int)
\end{lstlisting}

Because of the fact that there's no way to know in advance how many data type
declarations are there in source file I had to devise some way to assign a
unique tag for each of constructors that is not a built-in one. This is
precisely what the tagging phase is for. It works in two parts:

\begin{itemize}
  \item First it traverses the list of data types and its constructors in order
    to create a mapping between constructor name and its unique tag.
  \item Next the whole AST is recursively traversed and each
    \texttt{EConstrName} node is substituted for \texttt{EConstr} containing
    the previously associated tag and arity.
\end{itemize}

The definitions for those functionalities has been placed in
\textit{AbstractDataTypes.hs} file. This module also contains functionalities
responsible for querying constructors for tags, names, arities as well as
finding constructors given the \texttt{DataType} instance.


\section{Pattern Matching}
This section describes the details and motivation behing pattern matching as
well as describes details of algorithms that transform high-level pattern
matching constructs into enriched lambda calculus.

\hspace*{-1.5in}
\begin{lstlisting}[style=haskell,label=lst:length,caption={Calculating length of list.}]
length [] = 0;
length (x : xs) = length xs + 1

length Nil = 0;
length (Cons x xs) = length xs + 1
\end{lstlisting}

Listing~\ref{lst:length} reminds the already well known function for
calculating the length of a list, and also presents the version of the same
function but without \textit{syntactic sugar}. For the purpose of further
discussion lets assume that we're calculating the length of 3-element list:

\hspace*{-1.5in}
\begin{lstlisting}[style=haskell]
[1, 2, 3]
\end{lstlisting}
which is the same as:

\hspace*{-1.5in}
\begin{lstlisting}[style=haskell]
Cons 1 (Cons 2 (Cons 3 Nil))
\end{lstlisting}

The way the pattern matching compiler works, is that when length is applied to
an expression it might be not evaluated yet, so in order to proceed we have to
evaluate the function argument. Next, compiler checks whether argument is a Nil
constructor. This is not the case so compiler proceeds to checking against the
second branch. This time it succeeds and \texttt{x} is bound to \texttt{1} and
\texttt{xs} is bound to \texttt{[2, 3]}. The left-hand-sides of function
equations are called \textit{patterns}. They can consist of constructors,
variables, numbers, single characters and strings\footnote{Both the built-in
ones as well as those defined by programmer.}. In previous example we've seen
the combination of those, i.e. The pattern was composed of constructors and
variables. Nothing prevents programmer from creating the so called
\textit{nested patterns}. Example of such situation is present in
Listing~\ref{lst:second_element}, where the function is extracting the second
element from the list.

\hspace*{-1.5in}
\begin{lstlisting}[style=haskell,label=lst:second_element,caption={Greatest common divisor.}]
second [] = 0;
second [x] = 0;
second (x1 : (x2 : xs)) = x2
\end{lstlisting}

Patterns in the previous example were not \textit{overlapping}. It
means that changing the order of function equations would not alter the
behaviour of the function. The situation changes when we look at example
presented in Listing~\ref{lst:gcd}.

\hspace*{-1.5in}
\begin{lstlisting}[style=haskell,label=lst:gcd,caption={Greatest common divisor.}]
gcd a 0 = a;
gcd a b = gcd b (b % a)
\end{lstlisting}

Here if we've changed the order of equations, the function will never terminate
because the first clause would always match. This kind of patterns are called
\textit{overlapping patterns}.

Pattern matching can occur not only in function equations but also in source
code constructs. It can be used in \texttt{case} expressions as well as in
\texttt{let} and \texttt{letrec} bindings. Examples of those two pattern
matching uses are presented in Listing~\ref{lst:case_pattern_matching} and
~\ref{lst:letrec_pattern_matching}.  Making pattern matching available in
let(rec)s at first sight might look like very similar to other constructs. In
reality, however, it requires quite a complicated additional pass that
transform let(rec)s into more suitable form.  This transformation is described
in more detail in Section~\ref{sec:letrec_transform}.

\hspace*{-1.5in}
\begin{lstlisting}[style=haskell,label=lst:case_pattern_matching,caption={Pattern matching in
case expressions.}]
sqrList list =
    case list of
        [] -> []
        (x : xs) -> ((x * x) : sqrList xs)
\end{lstlisting}

\hspace*{-1.5in}
\begin{lstlisting}[style=haskell,label=lst:letrec_pattern_matching,caption={Pattern matching
  in let(rec) expressions.}]
let
    (x : xs) = [1, 2, 3, 4];
    (y : ys) = [5, 6, 7, 8]
in
    x + y
\end{lstlisting}

\subsection{Compiling Pattern Matching}
The simplest way to think of pattern matching is to match expression with first
pattern and in case it doesn't match, try the following patterns. Within each
equation patterns are tested from left to right\footnote{In case of functions
with multiple patterns}. Lets consider an example of \texttt{map} function from
Listing~\ref{lst:map}.

\hspace*{-1.5in}
\begin{lstlisting}[style=haskell,label=lst:map,caption={Map function example.}]
map f [] = [];
map f (x : xs) = ((f x) : (map f xs));

main = map (\x . x * x) [1, 2, 3, 4, 5]
\end{lstlisting}

The \texttt{map} function returns a list constructed by appling a function
\texttt{f} to all items in a list passed as the second argument. What our
mental model of pattern matching will do when evaluating the \texttt{main}
function, is that it will first try to match the \texttt{(\textbackslash x . x
* x)} lambda abstraction with \texttt{f} variable which will succeed. Matching
any expression to variable will result in success. Then the list \texttt{[1, 2,
3, 4, 5]} will be matched against an empty list - \texttt{[]} and it will fail.
In this situation, according to our simple strategy the algorithm will try to
match the second equation. So it will again try to match the lambda abstraction
with \texttt{f} variable pattern. As we can observe evaluation according to the
simple algorithm might result in redundant computations, and in case of more
complicated patterns it might be a substantial drawback. In order to be able to
perform pattern matching in more optimized fashion we have to transform into
the following form:

\hspace*{-1.5in}
\begin{lstlisting}[style=haskell,mathescape=true]
map = $\lambda$f.$\lambda$xs.
    case xs of
        Nil       -> Nil
        Cons y ys -> Cons (f y) (map f ys)
\end{lstlisting}

The remaining part of this section contains a description of an algorithm that
is capable of transforming pattern matching into the form that uses case
expressions instead of \textit{pattern-matching lambda abstractions}, thus allowing to
compile patterns more efficiently. In order to see a more detailed analysis
refer to Chapters 4 and 5 of \cite{Jon87}.

\subsection{Pattern-matching Lambda Abstractions}
In lambda calculus, regular lambda abstraction is an expression of the
following form:

\hspace*{-1.5in}
\begin{lstlisting}[style=haskell,mathescape=true]
$\lambda x.t$
\end{lstlisting}
where \textit{x} represents a \textit{variable} and \textit{t} stands for
\textit{term}. Intuitively, a lambda abstraction is an anonymous function that
takes a single input, and the \(\lambda\) is said to bind \textit{x} in
\textit{t}, and an application \textit{ts} represents the application of input
\textit{s} to some function \textit{t}. In the lambda calculus, functions are
taken to be first class values, so functions may be used as the inputs to other
functions, and functions may return functions as their outputs.

The difference between regular and pattern-matching lambda abstractions lies in
a fact that instead of variable $x$ on the left side a pattern might be
used. So it might look like that:

\hspace*{-1.5in}
\begin{lstlisting}[style=haskell,mathescape=true]
$\lambda p.t$
\end{lstlisting}
where $p$ is a pattern\footnote{Variable is also considered as a
pattern, so regular lambda abstractions form a subset of pattern-matching ones.}

\subsection{Free variables}
\label{sec:free_variable}

For the purpose of further deliberations I'll present the definition of
\textit{free variables}:
\begin{definition}
  \label{def:free_variable}
  A free variable is a notation that specifies places in an expression where
  substitution may take place. The idea is related to a placeholder (a symbol
  that will later be replaced by some literal string), or a wildcard character
  that stands for an unspecified symbol.
\end{definition}

In computer programming, a free variable is a variable referred to in a
function that is not a local variable nor an argument of that function:

\subsection{Translating function equations to pattern-matching lambda
abstractions}
For the purpose of further explanations lets consider the following function
definition:

\hspace*{-1.5in}
\begin{lstlisting}[style=haskell,mathescape=true]
$f\:p_{1} = e_{1}$
$f\:p_{2} = e_{2}$
   $\vdots$
$f\:p_{n} = e_{n}$
\end{lstlisting}

The pattern matching will be executed from top to bottom until some of them
matches or none of them matches. In the first situation the value of an
expression on the right hand side of matched pattern should be returned and
none of the following patterns should be matched against. In the second one an
error should be reported that no matching branch was found. With this in mind
we can translate the definition of $f$ into following one:

\hspace*{-1.5in}
\begin{lstlisting}[style=haskell,mathescape=true]
f = $\lambda$x.((($\lambda p_{1}'$.$e_{1}'$) x) []
        (($\lambda p_{2}'$.$e_{2}'$) x) []
              $\vdots$
        (($\lambda p_{n}'$.$e_{n}'$) x) []
        Error
\end{lstlisting}
Here, $x$ is a newly introduced variable name to temporarily hold the argument
passed to $f$. The $x$ must not be free in any $e_{n}$ in terms of free
variable definition. Apostrophe means that an expression or pattern has been
translated too. The $[]$ represents an infix function that chooses the next
pattern to match against in case of current pattern's failure.

Lets see our translation in action taking as an example the \texttt{map}
function:

\hspace*{-1.5in}
\begin{lstlisting}[style=haskell]
map f [] = [];
map f (x : xs) = ((f x) : (map f xs))
\end{lstlisting}
The result of such translation would be following:

\hspace*{-1.5in}
\begin{lstlisting}[style=haskell,mathescape=true]
map = $\lambda$f.$\lambda$xs.((($\lambda$Nil.Nil) xs) []
              (($\lambda$(Cons y ys).(Cons (f x) (map f ys))) xs) []
              Error
\end{lstlisting}
In this example $PatternMatchingError$ will never be returned because the set
of constructors $Nil$ and $Cons$ covers the whole domain of constructors for
$List$ data type. It is possible however to construct an example that will cause
that error to occur.

\subsection{Pattern matching algorithm}
\label{sec:pattern_matching_algorithm}

In general a multi-pattern function definition in Kivi of the form:

\hspace*{-1.5in}
\begin{lstlisting}[style=haskell,mathescape=true]
f $p_{1,1}$ $p_{1,2}$ $\ldots$ $p_{1,n}$ = $e_{1}$
f $p_{2,1}$ $p_{2,2}$ $\ldots$ $p_{2,n}$ = $e_{2}$
          $\vdots$
f $p_{m,1}$ $p_{m,2}$ $\ldots$ $p_{m,n}$ = $e_{m}$
\end{lstlisting}
is going to be translated into:

\hspace*{-1.5in}
\begin{lstlisting}[style=haskell,mathescape=true]
f = $\lambda v_{1}$.$\lambda v_{2}$ $\ldots$ $\lambda v_{n}$.(($\lambda p_{1,1}'$.$\lambda$.$p_{1,2}'$ $\ldots$ $\lambda p_{1,n}'$) $v_{1}$ $v_{2}$ $\ldots$ $v_{n}$) []
                 (($\lambda p_{2,1}'$.$\lambda$.$p_{2,2}'$ $\ldots$ $\lambda p_{2,n}'$) $v_{1}$ $v_{2}$ $\ldots$ $v_{n}$) []
                                 $\vdots$
                 (($\lambda p_{m,1}'$.$\lambda$.$p_{m,2}'$ $\ldots$ $\lambda p_{m,n}'$) $v_{1}$ $v_{2}$ $\ldots$ $v_{n}$) []
                 Error
\end{lstlisting}

We've arrived to the point where the pattern matching algorithm starts its
work. The goal is to translate this form into one that uses \texttt{case}
expressions. The code responsible for this task has been placed in
\textit{PatternMatching.hs} module and will be used as a reference. The heart
of algorithm is formed by \texttt{match*} set of mutually recursive functions,
specifically \texttt{matchSc} is the one that performs pattern matching on
supercombinators, and \texttt{matchEquations} matches function definition
equations. \texttt{matchEquations} has the following type signature:

\hspace*{-1.5in}
\begin{lstlisting}[style=haskell]
matchEquations :: NameSupply
               -> [DataType]
               -> Int
               -> [Name]
               -> [Equation]
               -> Expr Pattern
               -> (NameSupply, Expr Pattern)
matchEquations ns dts n vs eqs def
\end{lstlisting}
where
\begin{itemize}
  \item \texttt{ns}: Name supply for generating new unique names for
    introduced variables. The core logic is implemented in
    \textit{NameSupply.hs} module. It provides a generic facility to generate
    unique names and is used across many modules.
  \item \texttt{dts}: A set of defined data types needed later in constructor
    rule (see ~\ref{sec:constructor_rule}). As shown below \texttt{DataType}
    instances consist of name and list of \texttt{Constructor} instances.
    \texttt{Constructor}s in turn are built from name, tag and arity:

\hspace*{-1.5in}
\begin{lstlisting}[style=haskell]
type DataType = (Name, [Constructor])
type Constructor = (Name, Int, Int)
\end{lstlisting}

  \item \texttt{n}: The length of the list of argument variables
  \item \texttt{vs}: A list of argument variables. These names are generated
    with the help of \texttt{NameSupply} instance.
  \item \texttt{eqs}: Function equations. They consist of list of patterns as
    well as expression body:

\hspace*{-1.5in}
\begin{lstlisting}[style=haskell]
type Equation = ([Pattern], Expr Pattern)
\end{lstlisting}
\texttt{Expr Pattern} is a data type for representing expressions that contains
patterns. It uses a parametrized type \texttt{Expr a} where the parameter
\texttt{a} represents type that can be used as function arguments, left-hand
sides of let(rec) bindings, etc. It was previously described in
Section~\ref{sec:syntax_analysis}.
To give an example of how such expression with patterns might look like, let me
show you this \texttt{case} expression:

\hspace*{-1.5in}
\begin{lstlisting}[style=haskell]
case [1] of
    []       -> 0;
    (x : xs) -> x
\end{lstlisting}
Syntax analyser will transform this source code into the following internal AST
representation\footnote{\texttt{EConstr 2 0} represents a \texttt{Nil}
constructor, whereas \texttt{EConstr 3 2} stands for \texttt{Cons} constructor.}:

\hspace*{-1.5in}
\begin{lstlisting}[style=haskell]
ECaseConstr ((EConstr 3 2 (ENum 1)) (EConstr 2 0))
            [(PConstr 2 0 [], ENum 0),
             (PConstr 3 2 [PVar x, PVar xs], EVar x)]
\end{lstlisting}

\texttt{Pattern} type instances on the other hand represents patterns itself.
Patterns might be numbers, variables, constructors or special entities standing
for patterns chosen when none of other branches match. Below, the definition of
\texttt{Pattern} type is presented:

\hspace*{-1.5in}
\begin{lstlisting}[style=haskell]
data Pattern = PNum Int
             | PVar Name
             | PChar Int
             | PConstrName Name [Pattern]
             | PConstr Int Int [Pattern]
             | PDefault
\end{lstlisting}


  \item \texttt{def}: The default expression to evaluate when none of patterns
    matches
\end{itemize}

\texttt{matchEquations} function work according to four rules: the variable,
constructor, mixture and empty ones. The problem of determining which is
applicable given the current state, is solved by looking at first elements of
equations list. Depending on the type of currently considered equation, the
corresponding rule is applied. Helper function \texttt{classifyEquation}
returns the type of equation based on its first element. Following sections
describe each rule in more detail, as well as present a few hopefully
clarifying examples.

Examples in the rest of this section contains a simplified calls to
\texttt{matchEquations} function, where only \texttt{vs}, \texttt{eqs} and
\texttt{def} arguments are presented. The rest of them plays rather a minor
role in understanding the behaviour of algorithm.


\subsection{The Variable Rule}
This rule applies when every equation begins with a variable pattern. In such
case it is possible to remove that variable from every equation and also remove
the corresponding argument variable from \texttt{vs} list. After that we should
transform every expression corresponding to that pattern into one, where every
occurence of removed variable is substituted with the corresponding variable
from \texttt{vs} list. This behaviour is implemented by \texttt{matchVar}
function.
Our previous example, the \texttt{map} function eventually runs into a situation
where variable rule is applicable\footnote{This is a simplified call to
\texttt{matchEquations} where not all arguments are present}:

\hspace*{-1.5in}
\begin{lstlisting}[style=haskell,mathescape=true]
matchEquations [$v_{1}$, $v_{2}$]
               [([f, Nil], Nil),
                ([f, Cons y ys], Cons (f y) (map f xs))]
               Error
\end{lstlisting}
After applying the variable rule it would have such form:

\hspace*{-1.5in}
\begin{lstlisting}[style=haskell,label=lst:variable_rule, caption={State after applying
  variable rule.}, mathescape=true]
matchEquations [$v_{2}$]
               [([Nil], Nil),
                ([Cons y ys, Cons ($v_{1}$ y) (map $v_{1}$ xs)])]
               Error
\end{lstlisting}

\subsection{The Constructor Rule}
\label{sec:constructor_rule}
Similarly constructor rule is applied when every equation begins with
constructors. Equations are groupped together according to the constructor and
a case expression is introduced in order to choose the right branch. Each group
is now considered separately by recursive calls to \texttt{matchEquation}. New
argument variables are added to each group \texttt{vs} variable according to
the arity of a constructor. This logic has been implemented in
\texttt{matchConstr} function. Again a simple example will clarify this step.

After applying variable rule we've reached the state presented in
Listing~\ref{lst:variable_rule}. Here there are two constructors for two
equations: \texttt{Nil} and \texttt{Cons} so constructor rule applies:

\hspace*{-1.5in}
\begin{lstlisting}[style=haskell,label=lst:constructor_rule, caption={State after applying
  constructor rule.}, mathescape=true]
case $v_{2}$ of
    Nil       -> matchEquations []
                                []
                                Error
    Cons $v_{3}$ $v_{4}$ -> matchEquations [$v_{3}$, $v_{4}$]
                                [y, ys]
                                Error
\end{lstlisting}

\subsection{The Mixture Rule}
The need for additional rule arises from the case when neither all equations
begin with variable nor with constructor, hence the name - mixture rule. Here
is an example of function that causes such situation to occur:

\hspace*{-1.5in}
\begin{lstlisting}
length (x : xs) = 1 + length xs;
length empty = 0
\end{lstlisting}
After transforming this function we'll get:

\hspace*{-1.5in}
\begin{lstlisting}[style=haskell,label=lst:mixture_length, mathescape=true,
  caption={Mixture rule application.}]
matchEquations [$v_{1}$]
               [([Cons x xs], 1 + length xs),
                ([Nil], 0)]
\end{lstlisting}
When such situation occurs, pattern matching algorithm partitions the set of
equations into subsets in a way that each partition contains only elements of
the same type, i.e. In each partition there are only elements that are either
constructors or variables or numbers, etc. This logic is implemented by
\texttt{partition} function in \textit{Utils.hs} module. Once the partitioning
is done it's easy to work out other details. The \texttt{matchEquations} call
from Listing~\ref{lst:mixture_length} is equivalent to following one:

\hspace*{-1.5in}
\begin{lstlisting}[style=haskell,mathescape=true]
matchEquations [$v_{1}$]
               [([Cons x xs], 1 + length xs)]
               (matchEquations [$v_{1}$]
                               [([Nil], 0)]
                               Error)
\end{lstlisting}
After evaluating each of the \texttt{matchEquations} calls the length function
will look like this:

\hspace*{-1.5in}
\begin{lstlisting}[style=haskell,mathescape=true]
length = $\lambda v_{1}$
    case $v_{1}$ of
        Cons $v_{1}$ $v_{2}$ -> 1 + length $v_{2}$
        Nil       -> 0
\end{lstlisting}


\subsection{The Empty Rule}
Eventually, after applying the previous rules, our algorithm will run into a
situation where the variable list \texttt{vs} is empty, such as the one at
Listing~\ref{lst:constructor_rule} in the case for \texttt{Nil} constructor:

\hspace*{-1.5in}
\begin{lstlisting}[style=haskell]
matchEquations [] [] expr
\end{lstlisting}
When algorithm arrives into such state, the result is equal to \texttt{expr}.

\subsubsection{Summary}
According to the rules above one can transform every pattern-matching lambda
abstraction to a corresponding (and semantically identical) \texttt{case}
expression. This follows directly from the description of rules.

\section{Transforming \texttt{let(rec)} expressions}
\label{sec:letrec_transform}
The reason why we need to perform \texttt{let(rec)} transformations is due to
the fact that letrec binders\footnote{Binders of a \texttt{let(rec)}
expressions are the left-hand sides of each definition} might contain arbitrary
patterns in them. This prevents us from using the machinery that would be used
instead, if those binders were allowed to be simple variables only. This
section is dedicated to show how the transformation from enriched version of
\texttt{let} and \texttt{letrec} bindings to a simple lambda calculus work in
Kivi. Code implementing functionalities from this section has been placed in
\textit{LetTransformer.hs} module.

The rest of this chapter is devoted to presenting transformations which
translate a program into one, in which only \textit{simple let(rec)}s are
present. Simple \texttt{let(rec)}s are such, that their binders contain only
patterns that are either variable patterns or simple patterns(numbers,
characters, strings, etc.).

The first step of this process is to perform a \textit{conformality
transformation} described in Section~\ref{sec:conformality_transformation}.
Later, general \texttt{let} expressions, are converted into simple
\texttt{let}s via irrefutable \texttt{let}s, obtained by previous
transformation. On the other hand general \texttt{letrec}s are first
transformed into irrefutable \texttt{letrec}s and finally into simple
\texttt{letrec}s. This process is shown in Figure~\ref{fig:letrec_transform}:

\begin{figure}[h!]
  \centering

  TODO: diagram pokazujacy w jakiej kolejnosci sa aplikowane transformacje.

  \caption{Transformation of \texttt{let(rec)} expressions.}
  \label{fig:letrec_transform}
\end{figure}

\subsection{Refutable and irrefutable patterns.}
\label{sec:irrefutable_patterns}
We can distinguish two different types of \texttt{let(rec)} bindings:
\textit{Refutable} and \textit{irrefutable} ones. Let's look at this sample
\texttt{let} expression:

\hspace*{-1.5in}
\begin{lstlisting}[style=haskell,mathescape=true,label=lst:conformality_check,caption={Pattern matching \texttt{let} binding.}]
let (Cons x xs) = $expr$ in ...
\end{lstlisting}
The problem arises because \texttt{expr} might be evaluated to \texttt{Nil},
and in such situation this pattern matching will fail. This is precisely the
characteristic that distinguishes between the two aforementioned types of
\texttt{let(rec)} bindings. In short refutable bindings are those which contain
at least one refutable pattern and thus may fail during evaluation. On the
other hand irrefutable ones are those in which a failure cannot occur, thus
when all patterns are irrefutable.

\begin{definition}
  \label{def:irrefutable_pattern}
  A pattern is irrefutable if any of these conditions apply:
  \begin{itemize}
    \item It is a simple pattern (i.e. number pattern, character pattern, etc.)
    \item It is a variable pattern
    \item It has the form ($c$ $p_{1}$ $p_{2}$ \ldots $p_{n}$), where $c$ is
      constructor and $p_{1}$ $p_{2}$ \ldots $p_{n}$ are irrefutable patterns
  \end{itemize}
\end{definition}
The function \texttt{isRefutable} is implemented in exactly this fashion.

\subsection{Conformality transformation}
\label{sec:conformality_transformation}
A \textit{conformality transformation} is an activity to translate the program
into one that checks for patterns mismatch in \texttt{let(rec)} bindings. It
will contain only irrefutable \texttt{let(rec)} bindings. We say that such
program performs a \textit{conformality check}. Lets also mention that such
transformation is only needed in case of refutable bindings, so there's no need
to perform this, possibly expensive, computation in when patterns cannot fail.
So the first step of conformality transformation is partitioning
\texttt{let(rec)} bindings regarding to their refutability. Once we have all
patterns that might fail in one place we need to find a way to transform
\texttt{let(rec)}s into a form that conformality check is done.  Considering
our running example from Listing~\ref{lst:conformality_check} our compiler
would transform this expression into:

\hspace*{-1.5in}
\begin{lstlisting}[style=haskell,mathescape=true]
let (Cons x xs) = let $v_{1}$ = $expr$
                  in (($\lambda$p.(Cons x xs)) $v_{1}$) [] Error
\end{lstlisting}
where $v_{1}$ is newly introduced unique variable.

Based on that observation we could derive the general form of conformality
transformation from:

\hspace*{-1.5in}
\begin{lstlisting}[style=haskell,mathescape=true]
let $p_{1}$ = $e_{1}$;
    $p_{2}$ = $e_{2}$;
       $\vdots$
    $p_{n}$ = $e_{n}$
in $\ldots$
\end{lstlisting}
into:

\hspace*{-1.5in}
\begin{lstlisting}[style=haskell,mathescape=true]
let
    ($c_{k1}$ $v_{1,1}$ $\ldots$ $v_{1,k1}$) = let $w_{1}$ = $e_{1}$
                      in (($\lambda p_{1}$.($c_{k1}$ $v_{1,1}$ $\ldots$ $v_{1,k1}$)) $w_{1}$) [] Error
                               $\vdots$
    ($c_{kn}$ $v_{n,1}$ $\ldots$ $v_{n,kn}$) = let $w_{n}$ = $e_{n}$
                      in (($\lambda p_{n}$.($c_{kn}$ $v_{n,1}$ $\ldots$ $v_{n,kn}$)) $w_{n}$) [] Error
in $\ldots$
\end{lstlisting}
where:
\begin{itemize}
  \item $c_{ki}$ is a constructor of arity equal to the number of variables
    bound by pattern $p_{i}$. Lets call the set of such variables $Var(p_{i})$
  \item $\{v_1, v_2, \ldots, v_n\} = Var(p_i)$
  \item $w_{i}$ is distinct from every $v \in Var(p_i)$
\end{itemize}
The function which calculates the $Var(p_i)$ set is called
\texttt{getPatternVarNames} and its behaviour is works by collecting already
seen variable names while recursively traversing the pattern:

\hspace*{-1.5in}
\begin{lstlisting}[style=haskell]
getPatternVarNames :: Pattern -> [Name]
getPatternVarNames (PNum n) = []
getPatternVarNames (PChar c) = []
getPatternVarNames (PVar v) = [v]
getPatternVarNames (PConstr tag arity patterns) = foldl collectVars [] patterns
    where
        collectVars vars pattern = vars ++ getPatternVarNames pattern
\end{lstlisting}

The implementation of conformality transformation has been implemented as
\texttt{conformalityTransform} in \textit{LetTransformer.hs}.

\subsection{Transforming irrefutable \texttt{let}s into Simple \texttt{let}s}
As previously mentioned irrefutable \texttt{let} is such that contain only
irrefutable patterns as binders. There are three cases in which we consider a
pattern irrefutable according to Definition~\ref{def:irrefutable_pattern}. In
the first two patterns on the left-had side of definition might be either
variable or simple patterns(numbers, characters, etc.). In such case these
patterns are already simple, so there's nothing else to do. On the other hand
when patterns are irrefutable constructor patterns they are take the following
form:

\hspace*{-1.5in}
\begin{lstlisting}[style=haskell,mathescape=true]
  let ($c$ $p_1, p_2 \ldots, p_n$) = $expr$ in $\ldots$
\end{lstlisting}
where each $p_1, p_2, \ldots, p_n$ are irrefutable patterns. In order to
convert such expression into a simple \texttt{let}, we can apply the following
transformation\footnote{Without loss of generality we can assume that each
\texttt{let} contains only one definition. Every program could be easily
transformed to conform to this condition.}:

\hspace*{-1.5in}
\begin{lstlisting}[style=haskell,mathescape=true]
let $v$ = $expr$
in (let $p_1$ = $Select\mbox{-}r\mbox{-}1$ $v$;
        $p_2$ = $Select\mbox{-}r\mbox{-}2$ $v$;
              $\vdots$
        $p_n$ = $Select\mbox{-}r\mbox{-}n$ $v$
    in $\ldots$)
\end{lstlisting}
Here, $v$ is a newly introduced variable, $r$ is the arity of the $c$
constructor, whereas each $Select\mbox{-}r\mbox{-}i$ is a function that selects
the $i^{th}$ component of a constructor with arity $r$. Again a simple example
should clarify this:

\hspace*{-1.5in}
\begin{lstlisting}[style=haskell,mathescape=true]
let (Cons x xs) = (Cons 1 (Cons 2 (Cons 3 Nil)))
\end{lstlisting}
after applying the aforementioned transformation will be converted into:

\hspace*{-1.5in}
\begin{lstlisting}[style=haskell,mathescape=true]
let $v$ = (Cons 1 (Cons 2 (Cons 3 Nil)))
in (let x  = $Select\mbox{-}2\mbox{-}1$ $v$;
        xs = $Select\mbox{-}2\mbox{-}2$ $v$)
\end{lstlisting}
$Select\mbox{-}2\mbox{-}1$ when applied to \texttt{(Cons 1 (Cons 2 (Cons 3
Nil)))} will yield \texttt{1}, and $Select\mbox{-}2\mbox{-}2$ is going to return
\texttt{Cons 2 (Cons 3 Nil)}.

\subsection{Transforming irrefutable \texttt{letrec}s into Simple \texttt{letrec}s}
Transformation from irrefutable \texttt{letrec}s into simple ones is almost
identical to that concerning \texttt{let}s except from the fact that variables
from all definitions should be groupped together in order to ensure that
they're visible from other bindings. Here's the general scheme for such
conversion:

\hspace*{-1.5in}
\begin{lstlisting}[style=haskell,mathescape=true]
letrec ($c_1$ $p_{1,1}, p_{1,2} \ldots, p_{1,n1}$) = $e_1$
       ($c_2$ $p_{2,1}, p_{2,2} \ldots, p_{2,n2}$) = $e_2$
                $\vdots$
       ($c_k$ $p_{k,1}, p_{k,2} \ldots, p_{k,nk}$) = $e_k$
in $expr$
\end{lstlisting}
is going to be translated into following expression:

\hspace*{-1.5in}
\begin{lstlisting}[style=haskell,mathescape=true]
letrec
    $v_1$ = $e_1$;
    $p_{1,1}$ = $Select\mbox{-}n1\mbox{-}1$ $v_1$;
              $\vdots$
    $p_{n1,1}$ = $Select\mbox{-}n1\mbox{-}n1$ $v_1$;
    $v_2$ = $e_2$;
    $p_{2,1}$ = $Select\mbox{-}n2\mbox{-}1$ $v_2$;
              $\vdots$
    $p_{n2,1}$ = $Select\mbox{-}n2\mbox{-}n2$ $v_2$;
              $\vdots$
    $v_k$ = $e_k$;
    $p_{k,1}$ = $Select\mbox{-}nk\mbox{-}1$ $v_k$;
              $\vdots$
    $p_{nk,1}$ = $Select\mbox{-}nk\mbox{-}nk$ $v_k$;
in $expr$
\end{lstlisting}

\subsection{Summary}
After all these transformations took place, all binders in every
\texttt{let(rec)} expressions should become variable pattern in form
\texttt{PVar v = $expr$} or simple pattern. Now the pattern-matching algorithm,
described in Section~\ref{sec:pattern_matching_algorithm} can be used because
it was developed under assumption that left-hand sides of \texttt{let(rec)}
binders doesn't contain refutable and complex patterns. Therefore it is
important to run the conformality transform before pattern matching phase.

\section{Transforming \texttt{case} expressions}

\section{Dependency Analysis}
Dependency analysis is a process of splitting up \texttt{let(rec)} bindings
into minimial-sized groups and preferring \texttt{let} over \texttt{letrec}
where possible. The reason for this choice is because the non-recursive version
of binding can be implemented more efficiently. The fact that \texttt{let(rec)}
expressions are not always as optimal as they could, is twofold:
\begin{itemize}
  \item First, programmer might have ommited the fact that the
    \texttt{let(rec)} expression he implemented is not recursive at all.
  \item Previous phases\footnote{as well as passes that are not yet available
    in Kivi} might be unnecesarily pessimistic, for example \texttt{where}
    clauses are always transformed into recursive \texttt{let}s for simplicity
    reasons.
\end{itemize}
Consider the following Kivi binding as an illustration of that fact:

\hspace*{-1.5in}
\begin{lstlisting}[style=haskell,label=lst:letrec_dependency_example,caption={Example of
  \texttt{letrec} before dependency analysis.}]
letrec
    p       = 3;
    q       = 7;
    x       = p + q;
    fac n   = if (n == 0) 1 (n * fac(n-1));
    res     = fac x;
    gcd a b = if (b == 0) a (gcd b (a % b))
in
    gcd 24 18
\end{lstlisting}
After analysing the dependencies and splitting the letrec into minimal
subgroups the expression will become:

\hspace*{-1.5in}
\begin{lstlisting}[style=haskell,label=lst:letrec_after_anlysis,caption={\texttt{letrec}
  after performing dependency anlysis.}]
let
    p = 3
let
    q = 7
let
    x = 10
in letrec
    fac n = if (n == 0) 1 (n * fac(n-1))
in letrec
    gcd a b = if (b == 0) a (gcd b (a % b))
in let
    res = fac x
in
    gcd 24 18
\end{lstlisting}
This section provides an efficient algorithm to perform dependency analysis
using \textit{strongly connected components}.

\subsection{The Algorithm}
The algorithm consists of a few steps that will be described below. The source
code for this logic has been placed in \textit{DependencyAnalyser.hs} module
with \texttt{analyseDeps} function as an entry point. However in order to fully
understand further deliberations, it is necessary to introduce a couple of
definitions, that will be given througout this section.
\begin{definition}
A directed graph is a pair G = (V, E) of:
\begin{itemize}
  \item a set V, whose elementes are called vertices or nodes,
  \item a set E of ordered pairs of vertices, called arcs, directed edges, or
    simply edges.
\end{itemize}
\end{definition}

\subsubsection{Constructing Dependency Graph}
The first step of dependency analysis algorithm is to create a
\textit{dependency graph} for each each \texttt{letrec} construct. It is
defined as follows:
\begin{definition}
A dependency graph is a directed graph, where:
\begin{itemize}
  \item a set $V$ of vertices is the set of all variables bound by
    \texttt{letrec}
  \item there's an edge from vertex representing variable a to vertex
    representing variable b, if b is present in the right-hand side of
    binding for a. Formally speaking, if b occurs as a free variable
    (Definition~\ref{def:free_variable}) definition of a.
\end{itemize}
\end{definition}

To get a feeling how this applies in practice lets look at the dependency graph
build for our example from Listing~\ref{lst:letrec_dependency_example}:

TODO: graf dla tego przykladu

\subsubsection{Strongly Connected Components}
Next in order to group variables that depend on each other into one
\texttt{letrec} binding, we have to determine which variables are mutually
recursive. This is done by calculating the \textit{strongly connected
components} (or \textit{SCC}) of a dependency graph. According to Wikipedia the
definition of strongly connected components is following:

\begin{definition}
Directed graph is said to be strongly connected if there is a path from each
vertex in the graph to every other vertex. In particular, this means paths in
each direction: a path from a to b and also a path from b to a.
The strongly connected components of a directed graph G are its maximal
strongly connected subgraphs.
\end{definition}

In our running example the set of strongly connected components will look as
in the next picture:

TODO: obrazek z silnymi spojnymi skladowymi

\subsubsection{Topological Sorting}
After successfully determining the maximal sets of mutually dependant variable
bindings, it's time to know which of these groups depend on each other. In this
way bound variables will be in scope when evaluating the value of bindings that
depend on them. The way to perform this step is to consider each strongly
connected component as a single vertex, thus forming the graph that is acyclic,
and then to put these vertices in a \textit{topologically sorted} order. After
this step the situation in our example will be following:

TODO: obrazek po sortowaniu topologicznym


\subsubsection{Building a new \texttt{letrec}}
The last step is to create the \texttt{letrec} binding for each strongly
connected component in topological order. In our example the structure from
Listing~\ref{lst:letrec_after_anlysis} will be created.


\subsection{Implementation}
Regarding implementation issues, the heart of the code responsible for
aforementioned tasks is implemented in \texttt{analyseExpr} function in case
for \texttt{let(rec)} expressions. First of all, each AST expression is
augumented with its free variables. Based on that, function \texttt{getEdges}
folds over \texttt{letrec} bindings and creates edges of the dependency graph:

\hspace*{-1.5in}
\begin{lstlisting}[style=haskell]
getEdges edges (name, (rhsFree, rhs)) =
    edges ++ [(name, v) | v <- (Set.toList $ Set.intersection binderSet rhsFree)]
\end{lstlisting}

Next, \texttt{scc} function is invoked in order to find strongly connected
components of dependency graph. It does so by first computing, for each vertex
$x$, the set of vertices accessible (directly or indirectly) from that vertex
using forward edges, as well as sets of vertices accessible using backward
edges. By computing the intersection of these sets, we can effectively get
component of vertex $x$:

\hspace*{-1.5in}
\begin{lstlisting}[style=haskell]
scc :: Show a => Ord a => (a -> [a]) -> (a -> [a]) -> [a] -> [Set a]
scc ins outs vs = topSortedSccs
    where
        topSortedVs = snd $ dfs outs (Set.empty, []) vs
        topSortedSccs = snd $ spanDfs ins (Set.empty, []) topSortedVs
\end{lstlisting}

The rest of the \texttt{analyseExpr} function is meant to split the input
\texttt{letrec} expression according to the computed strongly connected
component set.

\section{Lambda Lifting}
This section provides a description of how lambda abstractions are implemented
in Kivi. Lambda abstractions are more commonly known as \textit{local function}
or \textit{anonymous function} definitions concepts from imperative languages.
As previosly I'm going to describe a program transformation, which when
applied, converts the program into a form where lambda abstracions become
global supercombinators. This transformation is known as \textit{lambda
lifting} or \textit{closure conversion}. The reader is encouraged to read the
Chapter 6 in \cite{JonLes00}, as well as Chapter 13 in \cite{Jon87} to
get additional information on this subject.

For the purpose of our \textit{lambda lifter} described here, I'm going to
introduce the annotated version of Abstract Syntax Tree, that is capable of
carrying additional information needed for transformations described in this
chapter. The type that implements this was called \texttt{AnnExpr} and has
been placed in \textit{LambdaLifter.hs}. Its full definition is following:

\hspace*{-1.5in}
\begin{lstlisting}[style=haskell,label=lst:annotated_expression]
type AnnExpr a b = (b, AnnExpr' a b)

data AnnExpr' a b = AVar Name
                  | ANum Int
                  | AChar Int
                  | AConstr Int Int
                  | AAp (AnnExpr a b) (AnnExpr a b)
                  | ALet IsRec [AnnDefn a b] (AnnExpr a b)
                  | ACase (AnnExpr a b) [AnnAlt a b]
                  | ACaseSimple (AnnExpr a b) [AnnAlt a b]
                  | ACaseConstr (AnnExpr a b) [AnnAlt a b]
                  | ALam [a] (AnnExpr a b)
                  | ASelect Int Int a
                  | AError String
    deriving Show

type AnnDefn a b = (a, AnnExpr a b)
type AnnAlt a b = (Int, AnnExpr a b)
data AnnScDefn a b = AnnScDefn Name [a] (AnnExpr a b)
type AnnProgram a b = [AnnScDefn a b]
\end{lstlisting}

As in the case for general \texttt{Expr} data type, each constructor is
responsible for creating a corresponding expression type. The \texttt{b} type
variable stands for additional information that the following passes will
produce. For instance, the \texttt{freeVars} pass described below, gathers the
free variables from the expression, so the \texttt{b} type will represent the
set of free variables in this expression. The full type in this case will have
the AnnExpr \texttt{AnnExpr Name (Set Name)} signature.

Lambda lifting is a way of eliminate free variables from local function
definitions. The elimination of free variables allows the compiler to hoist
lambda abstractions out of their surrounding contexts into a fixed set of
top-level supercombinators with extra parameter for each free variable. As an
example of this we'll analyse the following Kivi function that multiplies two
integers without using the multiplication operator (in a rather not the most
optimal fashion)

\hspace*{-1.5in}
\begin{lstlisting}[style=haskell]
mul 0 y = 0;
mul x 0 = 0;
mul 1 y = y;
mul x y =
    let sum a = y + a
    in sum (mul (x-1) y);
\end{lstlisting}

The \texttt{sum} lambda abstraction can be removed by introducting a new
\texttt{sum'} supercombinator, that takes an additional argument $t$ as shown at
the example below:

\hspace*{-1.5in}
\begin{lstlisting}[style=haskell,mathescape=true]
$sum'$ $t$ a = $t$ + a
mul 0 y = 0;
mul x 0 = 0;
mul 1 y = y;
mul x y =
    $sum'$ y (mul (x-1) y);
\end{lstlisting}

Lambda lifter will define such supercombinator for each lambda abstraction and
each of these supercombinators will take additional arguments that has been
free in the body of lambda abstraction before transformation.

The lambda lifting algorithm consists of four parts executed one after another:

\hspace*{-1.5in}
\begin{lstlisting}[style=haskell]
lambdaLift :: CoreProgram -> CoreProgram
lambdaLift (adts, scs) = (adts, collectScs . rename . abstract . freeVars $ scs)
\end{lstlisting}

I'm going to describe and provide details of how each of them has been
implemented in Kivi. The source file to look for implementation details, is
\textit{LambdaLifter.hs}.

\subsection{Annotating expressions with free variables}
The first phase of an algorithm is to annotate each Abstract Syntax Tree
expression with free variable occurencs contained in it. As we can see in the
type signature for function \texttt{freeVars} it transforms a list of
supercombinator definitions into the similar list\footnote{\texttt{AnnProgram}
type, according to Listing~\ref{lst:annotated_expression}, is just the list of
annotated supercombinator definitions}:

\hspace*{-1.5in}
\begin{lstlisting}[style=haskell]
freeVars :: [CoreScDefn] -> AnnProgram Name (Set Name)
\end{lstlisting}

More complicated code appears in the definition of function
\texttt{calcFreeVars} that has been called from \texttt{freeVars}. Its
arguments are:
\begin{itemize}
  \item \texttt{localVars} keeps variable names that has been defined in outer
    scopes. If, during traversing of AST such variable is used, it means that
    it s free variable of current expression.
  \item Expression to visit
\end{itemize}

The cases for simple expressions like \texttt{EVar} or \texttt{ENum} are easy,
the most complicated one occurs when calculating free vars for
\texttt{let(rec)} bindings. The code in this case works by calculating the free
variable set for each of the right-hand sides of bound variables. Then if it is
the recursive let, the free variable set consist of sum of these sets without
variables bound in current \texttt{letrec} binding. Otherwise it's simply the
sum of all free variables in right-hand sides.

\subsection{Abstracting free variables}
After each expression has the set of its free variables assigned, the algorithm
proceeds by abstracting out the free variables from each lambda abstraction and
introducting the \texttt{let} binding of the form:

\hspace*{-1.5in}
\begin{lstlisting}[style=haskell,mathescape=true]
(let sc = \$v_1$ $v_2$ $\ldots$ $v_n$ $x_1$ $x_2$ $\ldots$ $x_n$ . $expr$ in sc) $v_1$ $v_2$ $\ldots$ $v_n$
\end{lstlisting}

for $v_1$, $v_2$, $\ldots$, $v_n$ being free variables of lambda abstraction.
The code responsible for this is in the \texttt{abstractExpr} function in the
branch where \texttt{ALam} expressions are considered:

\hspace*{-1.5in}
\begin{lstlisting}[style=haskell]
abstractExpr (freeVars, ALam args expr) =
    foldl EAp sc $ map EVar freeVarsList
    where
        freeVarsList = Set.toList freeVars
        sc = ELet False [("sc", scBody)] (EVar "sc")
        scBody = ELam (freeVarsList ++ args) (abstractExpr expr)
\end{lstlisting}

The example of how this code would perform on a real case might be as the one
below. The lambda abstraction:

\hspace*{-1.5in}
\begin{lstlisting}[style=haskell]
(\x . y * y + x * z + tmp)
\end{lstlisting}

will be transformed into:

\hspace*{-1.5in}
\begin{lstlisting}[style=haskell]
(let sc = (\y z tmp x . y * y + x * z + tmp) in sc) y z x
\end{lstlisting}

\subsection{Renaming supercombinators}
The next step performs the renaming of variables, effectively giving unique
name to each of the \texttt{sc} variables introduced by \texttt{let} bindings
in previous step. Given that introduced supercombinators have unique names it
would be possible to hoist them to the level of global supercombinators during
next phases of algorithm. This is implemented as \texttt{rename} function. It
calls a generic renaming function \texttt{renameGen}. \texttt{renameGen} takes
as an input a function that generate new unique suffixes and concatenate them
with old variable names. It also requires that a list of supercombinator
definitions is given. It works by recursively visiting each AST expression and
performing a rename where applicable. Example function that is supposed to
rename variables in \texttt{EVar} expression:

\hspace*{-1.5in}
\begin{lstlisting}[style=haskell]
renameExpr newNamesFun mapping ns (EVar v) =
    (ns, EVar v')
    where
        v' = case Map.lookup v mapping of
            (Just x) -> x
            Nothing -> v
\end{lstlisting}

Here, \texttt{mapping} is an instance of a \texttt{NameSupply} type. It is used
to carry around an information about already used names in form of map between
old variable name to new variable name. If currently considered variable is
already present in a map, it means that we should use the name saved there.

As a result \texttt{renameGen} and \texttt{rename} return supercombinators with
each variable name being unique.

\subsection{Collecting supercombinators}
Finally, in the last step, we can perform the 'lift part' of lambda lifting
algorithm. In this pass the supercombinators introduced earlier are collected
and raised to the top level.  Regarding source code, the only complicated bit
is in the \texttt{collectExpr} function when collecting supercombinators from
\texttt{let(rec)} expressions.  There are three possibilities where we should
collect supercombinators from:

\begin{itemize}
  \item Firstly, supercombinators might be nested in \texttt{let(rec)}
    expression. They are gathered by calling \texttt{collectExpr} on that
    expression and are stored in \texttt{exprScs} variable.
  \item Supercombinators might also be present directly as \texttt{let(rec)}
    bindings as $sc$ \texttt{=} $\lambda v_1$ $\ldots$ $v_n$ . $expr$.
    Therefore what Kivi does, is splitting the list of bindings into ones that
    are simple variable bindings and supercombinators.
  \item The third and the last possibility is when supercombinators are nested
    in the right-hand side of binding. Then we should extract them by means of
    calling the \texttt{collectExpr} on the right-hand side of binding
    definition. This is implemented by folding over bindings with
    \texttt{collectDef}.
\end{itemize}



\section{Lazy Lambda Lifting}
\section{Type Checker}


\chapter{The G-machine}
The G-machine is an implementation of virtual machine based on the idea of
\textit{graph reduction} that will be described in
Section~\ref{sec:graph_reduction}. It works by compiling supercombinators into
intermediate code called \textit{G-code} and later translating it further or
executing in an interpreter.

The advantage of choosing the intermediate representation instead of compiling
to a particular machine code lies in a fact that it would require rewriting of
the whole compiler if later decided to change the machine. Moreover, one of the
main characteristics of intermediate representations is the fact that they
should resemble the original source language, and generating the machine code
from it, should remain simple as well. Thus, translating the original source
code first into the G-code and later to the machine of our choice shall be much
simpler than doing it in a straight-forward manner. The third argument for
preferring having intermediate representation over a concrete one is that it
will give us the ability to create virtual machine interpreters or compilers
for any particular computer architecture we desire, by means of only switching
the code generator. Code generators will then need to transform the
intermediate language into machine code of that machine. This will greatly
simplify the development of such tools, because the issues of compiling
supercombinators to sequential code and compiling to machine code will be
clearly separated.  Thanks to these facts I was able to easily create two
implementations of the G-machine:

\begin{itemize}
  \item The first one, is a G-machine interpreter. It is particularly suited
    for debugging purposes and it might be used as a Kivi shell in future. It
    was created in Haskell, and described in Section~\ref{sec:interpreter}
  \item Another one generates the intermediate code for Low Level Virtual
    Machine, which later is compiled to machine code by LLVM compilers. Its
    description is the content of the Section~\ref{sec:llvm_codegen}.
\end{itemize}

\section{Graph reduction}
\label{sec:graph_reduction}
As previously stated the output of a syntax analyser is an Abstract Syntax
Tree. In above chapters we have applied a several transformations the the tree
so as to either allow special constructs like lambda abstractions or to
translate the source of the program to enriched lambda calculus. Eventually the
output of these transformations is still AST. \textit{Graph reduction} is an
implementation of efficient, non-strict supercombinator evaluation strategy.
Non-strictness in this context means that function arguments are evaluated on
demand. This is also known as \textit{lazy evaluation}. Another feature of
G-machine implementation is that values should only be evaluated once, in order
not to repeat the same calculations over and over.

\subsection{The Graph}
The abstract syntax tree is transformed to a graph due to conversions that take
place during graph reduction. Such graph is being stored in memory,
specifically it's kept on the heap. The nodes of the graph are dynamically
allocated on the heap. They contain a \textit{tag} that uniquely identifies the
type of a node, as well as additinoal fields like values, in case of number
nodes, or links (addresses) to other nodes in case of application nodes for
instance. For example the abstract way of visualising application node might be
like in the figure below:

\begin{figure}[h!]
  \centering

  TODO: Example of application node.

  \caption{Application node on the heap.}
  \label{fig:application_node}
\end{figure}

On the other hand number nodes might be perceived as following:

\begin{figure}[h!]
  \centering

  TODO: Example of number node.

  \caption{Number node no the heap.}
  \label{fig:num_node}
\end{figure}

\subsection{Evaluation Strategy}
There are many possible evaluation strategies. Many imperative languages
chooses to follow the \textit{innermost evaluation}. In this strategy the
arguments to a function are evaluated before the function application. In Kivi,
similarly as in other lazy functional languages the outermost redex is
evaluated first, thus Kivi adapts the \textit{innermost evaluation strategy}.
Function arguments are evaluated only in case it's absolutely needed, so
function arguments that are not used in the body of function will never be
evaluated, thus saving computations.

In Kivi reduction starts by evaluating the body of \texttt{main}
supercombinator. The process of reduction consist of transforming a graph
according to the specified rules and eventually arriving to the state where it
cannot be reduced any more. Such state is called \textit{weak head normal form}
(or WHNF for short). We assume that an expression is in the WHNF in several
cases. First of all, when it is a number, character, or any other simple data
type it's considered to be in WHNF. Another situation where there is nothing
more to reduce is if the expression is in ($f$ $e_1$ $e_2$ $\ldots$ $e_n$)
form, where $f$ is lambda abstraction, built-in function or constructor and $n$
is less or equal the expected number of arguments for $f$.

The process of evaluation can be described in four points:

\begin{itemize}
  \item Finding the next reducible redex
  \item If there's no redex which can be evaluated, then stop
  \item Otherwise reduce the redex
  \item Update the root of the redex, to save the result of computations
\end{itemize}

\subsection{Finding next redex}
The first step is to find the node of a graph where to perform the next
reduction. As already mentioned, among all such nodes we'll be looking for the
outermost one. The most straight-forward way to find it, is to follow the left
branches of application nodes (as shown in Figure~\ref{fig:application_node})
saving the traversed node addresses on the runtime stack. There are two cases
when this process stops, either when the supercombinator or built-in function
node is encountered.

\subsection{Reduction}
Reduction depends on the type of node the algorithm encountered. There are a
few cases that might occur:

\subsubsection{Reducing supercombinators}
When supercombinator node is to be reducted, first thing what G-machine does,
is to check whether there are enough arguments on the stack. If it is not the
case, algorithm returns, as such application appears to be a \textit{partial
application} forming a closure. On the other hand, if the number of arguments
on the stack is greater or equal to the expected number the process continues.
The arguments to a supercombinator consist of a set of right children of
application nodes that we have visited during traversal\footnote{At this step,
the \textit{rearrangement} of arguments takes place. This activity will be
described in next chapters, but it's just an optimization so I'm not going to
introduce it right now, not to make unnecessary complications.}. The addresses
that form the stack, are called \textit{spine} of the expression, and the
process of following the left branches of application nodes is commonly known
as \textit{unwinding} the spine. The reduction of a supercombinator can be
looked at, as instantiating a supercombinator body with arguments substituted
for formal parameters (as defined in Section 2.1.4 of \cite{JonLes00}).

\subsubsection{Reducing built-ins}
It might be the case however that the outermost redex found is not a
supercombinator node, but rather a built-in operator or function. In such case,
before applying the builtin, the required arguments should be evaluated to
WHNF. Therefore when evaluator finds itself in a state when built-in function
is the currently encountered node, it has to save the current stack on the
\textit{dump} and recursively invoke itself on all arguments that are not in
WHNF. Of course it might happen that during evaluation of arguments, the
algorithm once again runs into a situation when another builtin is to be
evaluated. In such case, it saves the current stack again and performs another
recursive invokation of itself. Dump therefore, is formed by putting stacks on
each other, so it might be perceived as stack of stacks.

This situation described above occurs when evaluating the expression \texttt{(1
+ (sqr 2)) * 3}. The graph built for this expression will look like one in
Figure~\ref{fig:expression_graph}

\begin{figure}[h!]
  \centering

  TODO: graf dla wyrazenia (1 + (sqr 2)) * 3

  \caption{Graph created for expression \texttt{(1 + (sqr 2)) * 3}.}
  \label{fig:expression_graph}
\end{figure}

Because \texttt{*} operator is a builtin, it requires that all its arguments
are in WHNF before being able to continue. The algorithm puts the current stack
onto a dump and starts to evaluate the first argument. Shortly it will run into
a state that \texttt{+} is to be applied, but once again \texttt{sqr 2} is not
in a WHNF, so the process continues.

\subsection{Updating}
\label{sec:updating}
When reduction is complete, there is a need to save results of our
computations. This is called \textit{updating} the root of redex. Important
thing to notice is that if the redex is shared between two nodes of our graph,
the evaluation should be done only once. In order to achieve this we have to
introduce a new node type called \textit{indirection node}. Otherwise the
reduced node (result of computations) will have to be copied to the root of
the redex. In case of that node not being fully evaluated it brings a risk of
performing computations multiple times. Indirection nodes work like a pointer
to other node, effectively making the machine to use the values computed last
time redex was reduced. This will prevent it from performing the same
calculations more than once.

\subsection{Evaluation Example}
Example of evaluating expression (TODO: example expression) is presented
in Figure~\ref{fig:example_reduction}.

\begin{figure}[h!]
  \centering

  TODO: Obrazek ilustrujacy stos i zbudowane w ten sposob drzewo ewaluacji
  wyrazenia

  \caption{Example reduction process.}
  \label{fig:example_reduction}
\end{figure}

\section{Compiling supercombinators}
Compilation is a process of turning supercombinator body into a sequence of
G-machine instructions. These instructions when executed will construct an
instance of a supercombinator with formal parameters substituted with actual
function arguments. When running a concrete program, consisting of
supercombinator definitions, at first it will be translated into G-code, and this
process may be seen as \textit{compile time}. Later G-code is going to be
executed on a virtual machine, the G-machine. The latter may be regarded as
\textit{run time}.


\subsection{Instruction set}
In Kivi the set of G-code instructions available is presented as an excerpt
from \textit{Common.hs} module below:

\hspace*{-1.5in}
\begin{lstlisting}[style=haskell]
type GmCode = [Instruction]

data Instruction = Unwind
                 | Pushglobal Name
                 | Pushconstr Int Int
                 | Pushint Int
                 | Pushchar Int
                 | Push Int
                 | Mkap
                 | Update Int
                 | Pop Int
                 | Slide Int
                 | Alloc Int
                 | Eval
                 | Add | Sub | Mul | Div | Neg | Mod
                 | Eq | Ne | Lt | Le | Gt | Ge
                 | Pack Int Int
                 | CasejumpSimple (Assoc Int GmCode)
                 | CasejumpConstr (Assoc Int GmCode)
                 | Select Int Int
                 | Error String
                 | Split Int
                 | Print
                 | Pushbasic Int
                 | MkBool
                 | MkInt
                 | Get
\end{lstlisting}

Here, \texttt{GmCode} type represents the G-code and is a simple list containing
\texttt{Instruction} instances. To give a hopefully clarifying example,
consider following supercombinator definition:

\hspace*{-1.5in}
\begin{lstlisting}[style=haskell,label=lst:sc_to_compile,caption=Supercombinator to compile.]
calc x y = 3+x*y
\end{lstlisting}

After compiling it to G-machine instructions it will become:

\hspace*{-1.5in}
\begin{lstlisting}[style=haskell,label=lst:gcode_supercombinator,caption={Compiled
  supercombinator body.}]
Push 0
Pushglobal "*"
Mkap
Push 2
Push 1
Mkap
Pushglobal "v2"
Mkap
Eval
Update 3
Pop 3
Unwind
\end{lstlisting}

The \texttt{Push n} instruction causes the \texttt{n}$^{th}$ value from the
stack to be pushed to the top of stack. Simply put, it copies the
\texttt{n}$^{th}$ argument. When the evaluation of supercombinator begins, the
stack is rearranged so that function arguments are on the top of the stack.
Thus \texttt{Push 0} will copy the first function argument to the top of the
stack. \texttt{Pushglobal v} is meant to push the address of the \texttt{v}
built-in or supercombinator on the stack. \texttt{Mkap} pops two elements,
creates an application node on the heap and pushes its address. \texttt{Update
n} instruction implements the behaviour described in
Section~\ref{sec:updating}. The \texttt{n} argument represents the offset of
the root redex from the top of stack. \texttt{Pop n} on the other hand is
supposed to remove all elements, that supercombinator declared on the stack,
and leave the result as a top element. The process of executing G-code
instructions from Listing~\ref{lst:gcode_supercombinator} is shown in figure
below:

\begin{figure}[h!]
  \centering

  TODO: G-code execution

  \caption{Execution of supercombinator from Listing~\ref{lst:gcode_supercombinator}.}
  \label{fig:gcode_execution}
\end{figure}

\subsection{Compiler implementation}
Kivi's compiler is implemented as a set of \textit{compilation schemes} in
\textit{GmCompiler.hs} file. Each such compiliation scheme is applicable in
certain conditions. For example the scheme implemented by \texttt{compileA}
function, is chosen when compiling branches of case expression. The main data
type representing compiler is \texttt{GmCompiler} defined as below:

\hspace*{-1.5in}
\begin{lstlisting}[style=haskell]
type GmCompiler = CoreExpr -> GmEnvironment -> GmCode
type GmEnvironment = Assoc Name Int
\end{lstlisting}

\texttt{GmEnvironment} is a data type for keeping associations between variable
names and their offset from the stack top. So our compiler is expected to be
perceived as a function that given the abstract syntax tree, as well as
environment produces G-machine instructions. Compilation process starts from
\texttt{compile} function. It turns the abstract syntax tree of a program into
an initial state prepared for either interpretation or further compilation (into
LLVM intermediate representation, in case of Kivi):

\hspace*{-1.5in}
\begin{lstlisting}[style=haskell]
compile :: CoreProgram -> GmState
compile (dts, scs) = ([], initialCode, [], [], [], heap, globals, initialStats)
    where
        (heap, globals) = buildInitialHeap scs
\end{lstlisting}

Here, the \texttt{buildInitialHeap} is used to fold over each supercombinator,
create a node of \texttt{NGlobal} type for it on the heap. Other possible node
types that might occur on the heap are:

\hspace*{-1.5in}
\begin{lstlisting}[style=haskell]
data Node = NNum Int
          | NChar Int
          | NAp Addr Addr
          | NGlobal Int GmCode
          | NInd Addr
          | NConstr Int [Addr]
          | NMarked Node
\end{lstlisting}

whereas heap itself is a mapping from \texttt{Addr} to \texttt{Node}. The
\texttt{NMarked} node type is used in garbage collector, see
Section~\ref{sec:garbage_collection}, for details.

In order to compile an expression compilation schemes are recursively applied
in accordance to the type of expression. Look at the case when
\texttt{compileC} compilation scheme is generating code for
\texttt{EAp}\footnote{This expression type represents function applications.}
expression:

\hspace*{-1.5in}
\begin{lstlisting}[style=haskell]
compileC (EAp e1 e2) env =
    compileC e2 env ++
    compileC e1 (argOffset 1 env) ++
    [Mkap]
\end{lstlisting}

In order to follow further deliberations, recall that application nodes consist
of two subexpressions. First one is a function to be applied, whereas another
is an argument. What \texttt{compileC} does, is to first generate code for an
argument. At this point when the generated code is executed the argument will
be pushed on the top of the runtime stack. Therefore all variables that were
there, will be moved up by one stack cell. This is exactly what
\texttt{argOffset} function does. After generating code for function too, an
\texttt{Mkap} instruction is added to apply argument to it. Again a simple
example might be worth more that thousand words. Below I present the
compilation steps required for expression from Listing~\ref{lst:sc_to_compile}.


TODO: Example of compiling supercombinator from listing~\ref{lst:sc_to_compile}.


\section{Interpreter}
\label{sec:interpreter}
Interpreter is defined as a program that executes instructions written in
programming language. There is a variety of ways that interpreter can perform
evaluation of a program. Kivi takes the approach of first compiling the source
to intermediate language, called G-code, and later executing these
instrutions.\footnote{Other ways may include, for example, directly executing
source code instructions}.

The G-machine is a \textit{finite state machine}. It consists of finite number
of states as well as transitions between these states. At every moment of
G-machine execution the state can be described by following tuple:

\subsection{State}
The meaning of each element is summarized below:
\begin{itemize}
  \item \texttt{output} is a \texttt{GmOutput} instance. It contains the otput
    that program produced during execution.
  \item \texttt{code} represents the currently executed sequence of
    instructions. It is implemented as an instance of \texttt{GmCode}.
  \item \texttt{stack}, \texttt{GmStack} instance, implements the G-machine's runtime
    stack and keeps the spine of expression.
  \item \texttt{dump}, described above, is a collection \texttt{GmDumpItems},
    where:

    \hspace*{-1.5in}
    \begin{lstlisting}[style=haskell]
      type GmDumpItem = (GmCode, GmStack, GmVStack)
    \end{lstlisting}
    It contains the previous states of execution, that were suspended due to
    evaluation of built-in function arguments. The dump is going to be unwound
    when current instruction set depletes.
  \item the \texttt{GmVStack} is introduced as an optimization to perform
    arithmetic and logical operations using registers or stack, instead of creating
    heap nodes for them.
  \item Supercombinator's graph is being kept on istance of type
    \texttt{GmHeap}.
  \item \texttt{GmGlobals} contains global function and operator addresses that
    point to nodes on the heap
  \item \texttt{GmStats} is meant to save the statistics gathered during
    execution. It might be number of reductions, number of nodes allocated,
    etc.

\end{itemize}

Therefore the G-machine state is implemented as a 8-tuple in
\textit{Common.hs}. This module contains also a variety of functions to both
inspect or create new state on the basis of old one. When program execution
begins, the \texttt{compile} function creates an \textit{initial} state with
all elements set to default values:

\hspace*{-1.5in}
\begin{lstlisting}[style=haskell]
type GmState = ( GmOutput
               , GmCode
               , GmStack
               , GmDump
               , GmVStack
               , GmHeap
               , GmGlobals
               , GmStats )
\end{lstlisting}

The \texttt{eval} function is an entry point for interpreter. It analyses
whether current state is a final one and if this is not the case, dispatches
the execution to an appropriate routine, based on next instruction in the
\texttt{GmCode} state component. This implements the transitions between states.

\subsection{State transition rules}
In G-machine, for each instruction there exists an associated function that
contains the logic of that specific instruction. In literature it is described
as \textit{state transition rules}. Given current instruction from
\texttt{code} as well as state, transition rules create new state where some of
components are changed due to results of executing that instruction. For
example, state transition for \texttt{Push n} instruction states, that the
stack component of G-machine state should have its n$^{th}$ element copied to
the top. Other elements should remain unchanged:

\hspace*{-1.5in}
\begin{lstlisting}[style=haskell]
push :: Int -> GmState -> GmState
push n state =
    putStack stack' state
    where
        stack' = argAddr : stack
        argAddr = stack !! n
        stack = getStack state
\end{lstlisting}

Assuming that for each
instruction there exists a unique transformation producing an updated state,
this gives us a picture of fully deterministic finite automaton.

The reader is advised to refer to \cite{Jon87} and \cite{JonLes00} for full
description of every transition.

\section{LLVM Code Generator}
\label{sec:llvm_codegen}
Compilers usually consist of two parts called \textit{frontend} and
\textit{backend}. The role of a frontend is to parse the source file, perform
semantic analysis and emit the processed version of input program called
\textit{intermediate representation} or \textit{IR} for short. Up to now I have
described the frontend of Kivi compiler with G-code being its intermediate
representation. The role of a backend on the other hand is to process the IR,
optimize it, map values to machine registers and emit an assembly language as
text. The final pass is done by \textit{assembler} which translates the
assembly language into binary machine code.

There is a wide choice of how to implement the backend of a compiler. I could
either write the backend by hand, which would be way more complicated than the
frontend, or translate the G-code to an existing language, such as C for
instance. This is the approach taken initially by C++ authors, and it worked
well because C++'s semantics resembles that of C. Unfortunately this is not the
case in Kivi, because its semantics is not close enough to C to use a
straightforward translation. The most obvious reason for this is that C is not
garbage collected, whereas Kivi is. These were the reasons why I chose another
way of implementing Kivi's backend, that is to translate source to intermediate
representation of some compiler toolkit, such as LLVM.

\subsection{Low Level Virtual Machine}
The \textit{Low Level Virtual Machine} (\cite{website:llvm}) is a compiler
infrastructure providing a set of compiler and toolchain technologies, designed
to optimize the compile, link and run time of programs written in variety of
programming languages. Originally it was an experiment to explore dynamic
compilation strategies for programming languages, but since then it has grown,
and gathered a number of subprojects under its name. Therefore it may well
serve as a middle layer of a full compiler toolchain allowing to derive from
its advanced optimization techniques. It works by taking an IR code from a
compiler and producing an optimized IR. The newly produced intermediate
representation can be linked into machine-dependent assembly code for a target
platform.

\subsection{LLVM Intermediate Representation}
Intermediate representation of LLVM provides an independent instruction set and
type system. It aims to be a low-level and general enough, to make it an easy
mapping for many high-level languages. The existence of type information,
usually absent in low-level representations, makes it possible to perform a
range of optimizations which otherwise won't be possible. Each instruction is
in \textit{single static assignment} form, which means that every variable can
be assigned only once and is not allowed to be modified later. Full description
of LLVM IR instructions can be found at \url{http://llvm.org/docs/LangRef.html}.

\subsection{Representing G-machine state}
The LLVM G-machine implementation consists of similar parts as the previously
described interpreter one. The code and dump parts are rather straightforward
because code consist of LLVM IR instructions, whereas dump is represented by
LLVM call stack. The more interresting bits are stack and heap that are
described in the next sections.

\subsubsection{Stack implementation}
G-machine's stack is implemented in \textit{program.st}
template that aims to provide the global declarations needed later. It is
represented by a data area to hold the stack elements as well as the stack
pointer. At each point of thof memorye program the stack pointer, \texttt{@sp} points to
the top of the stack:

\hspace*{-1.5in}
\begin{lstlisting}[style=assembler]
@stack = global [1000 x i64*] undef
@sp = global i64 undef
\end{lstlisting}

The drawback of current implementation is that size of the stack has to be
declared statically, thus not allowing the stack to grow arbitrarily up to
memory size. In future, however I'm planning to reimplement it to allow dynamic
memory allocation, similarly as vectors in C++ STL. Stack operations are
defined as functions in the same file. Thus \texttt{push} and \texttt{pop}
stack functions are defined as follows:

\hspace*{-1.5in}
\begin{lstlisting}[style=assembler,caption={\texttt{push} and \texttt{pop} operations on the
  LLVM stack.}]
define void @push(i64* %addr) {
    ; store address on stack
    %n = load i64* @sp
    %ptop = call i64**(i64)* @getItemPtr(i64 %n)
    store i64* %addr, i64** %ptop
    ; increment stack pointer
    call void(i64*)* @incSp(i64* @sp)
    ret void
}

define i64* @pop() {
    %ptop = call i64**()* @getTopPtr()
    %addr = load i64** %ptop
    call void(i64*)* @decSp(i64* @sp)
    ret i64* %addr
}
\end{lstlisting}

\texttt{getItemPtr}, \texttt{getTopPtr}, \texttt{incSp} and \texttt{decSp} are
helper functions implemented in the same file.


\subsubsection{Heap implementation}
G-machine's heap is represented by a memory area in the machine's heap. LLVM
doesn't provide instructions to manage memory on the heap. Instead one has to
use \textit{libc} calls to achieve this functionality. Each heap node consist
of a \textit{tag} and \textit{data}. The role of a tag is to differentiate
between different types of nodes. For example \texttt{Unwind} G-machine
instruction performs tag analysis in order to decide how to proceed further.
This is done using LLVM IR's \texttt{switch} construct and illustrated in
listing below:

\hspace*{-1.5in}
\begin{lstlisting}[style=assembler]
%ptop = call i64**()* @getTopPtr()
%top = load i64** %ptop
%tag = load i64* %top

switch i64 %tag, label %otherwise
    [ i64 1, label %NUM_UNWIND
      i64 2, label %AP_UNWIND
      i64 3, label %GLOB_UNWIND
      i64 4, label %IND_UNWIND
      i64 5, label %CONSTR_UNWIND ]

NUM_UNWIND:
...
CONSTR_UNWIND:
...
\end{lstlisting}
Here, the top stack element is fetched first, next the tag of current cell is
retrieved and depending on it, a jump to the correct piece of code is
performed.

On the other hand data part of node cell contain fields that carry the nodes'
payload. Both tag and each field occupy 8 bytes of memory. It might seem
wasteful to allocate 8 bytes for tag, when there are only few of them possible,
but in the end it made the implementation simpler due to uniform layout of
cells. As an example we can consider the application node. It contains the tag
field with the value of 2 inside (defined by \texttt{@AP\_TAG}), as well as
addresses of two other nodes in the graph, representing function and argument.

Each type of node can be allocated on the heap using a corresponding function.
For example in order to allocate a number node one would use \texttt{hAllocNum}
defined as:

\hspace*{-1.5in}
\begin{lstlisting}[style=assembler,caption={Definition of function allocating number node on
  the heap.}]
define i64* @hAllocNum(i64 %n) {
    %ptr = call i8*(i64)* @malloc(i64 16)
    %ptag = bitcast i8* %ptr to i64*
    %pval = call i64*(i64*)* @getNumPtr(i64* %ptag)
    %numtag = load i64* @NUM_TAG
    store i64 %numtag, i64* %ptag
    store i64 %n, i64* %pval
    ret i64* %ptag
}
\end{lstlisting}

\subsection{Code Generation}
Until now I have discussed the building blocks of LLVM code generator, yet
haven't said anything about how translation from G-code instructions to LLVM IR
really works. In this chapter I'm going to fix that.

LLVM code generator is implemented in \textit{CodeGen.hs} file. First, function
\texttt{genLLVMIR} generates the intermediate representation which is later
saved to file by \texttt{saveLLVMIR}. However the real heavy-lifting is done in
\texttt{translateToLLVMIR} with the help of \textit{HStringTemplate} library
(\cite{website:hstring_template}). It is reponsible for generating code chunks
by feeding templates that reside in \texttt{templates/} directory with template
parameters.

Code generator is called for each supercombinator and fetches its instructions.
After that \texttt{translateToLLVM} folds over these instructions and pattern
match on instruction type. For each of them there's a separate routine that
seeks for the corresponding template and instantiate it with parameters
substituted for G-code instruction arguments. Such filled supercombinator
templates are gathered and injected into the program's global template which is
saved to a file on a disk. As an example consider the code that instantiates a
template for \texttt{Pushint n} instruction:

\hspace*{-1.5in}
\begin{lstlisting}[style=haskell]
translateToLLVMIR mapping templates ninstr (Pushint n) =
    (ir ++ [template'], ninstr + 1)
    where
        Just template = getStringTemplate "pushint" templates
        template' = setManyAttrib
                [("n", show n), ("ninstr", show ninstr)]
                template
\end{lstlisting}

The code on listing above fetches the correct template, and sets the required
attributes. Corresponding \texttt{Pushint} template is presented below:

\hspace*{-1.5in}
\begin{lstlisting}[style=assembler]
%ptag.$ninstr$ = call i64*(i64)* @hAllocNum(i64 $n$)
call void(i64*)* @push(i64* %ptag.$ninstr$)
\end{lstlisting}

The \textit{gaps} to be filled with arguments are surrounded with \texttt{\$}
signs. As expected, this template when instantiated will allocate the number
node on the heap and later push the address of that node on stack. Each defined
variable has unique suffix because of static single assignment feature of LLVM
IR.


\section{Garbage Collection}
\label{sec:garbage_collection}
\textit{Garbage collection} or \textit{GC}, as described in
\cite{wiki:garbage_collection} is a form of automatic memory management. It
collects and manages the memory that was allocated on the heap by running
program. By management I mean determining which objects in memory will not be
needed anymore, so they can be safely returned to the free memory pool.
Deciding upon which memory is not going to be used by the program anymore, is
one of the biggest issues in designing garbage collector. Another important
problem is to find the correct moment for GC to run.

\subsection{Mark and Sweep GC}
Current approach that has been taken in Kivi's garbage collection is to
implement a simple \textit{mark and sweep} algorithm. It's sources are
available in the \textit{Gc.hs} file. In this kind of garbage collection
strategy, each node on the heap, has an associated value that says, whether or
not, the current node is used. This is the role of \texttt{NMarked} node
constructor mentioned in one of previous chapters. If the node is wrapped into
this type it means it might still be needed during program execution.

Garbage collection is being triggered when the number of allocated heap nodes
reaches a certain, configurable threshold value. In such case, the entry point
function, \texttt{gc} is called:

\hspace*{-1.5in}
\begin{lstlisting}[style=haskell]
gc :: GmState -> GmState
gc state = putHeap heap' state
    where
        heap' = scanHeap $ foldl markFrom heap $ findRoots state
        heap = getHeap state
\end{lstlisting}

It performs a transformation between G-machine states, where the returned one
has its heap size reduced by freeing unused nodes. Algorithm is divided into
several parts:

\begin{itemize}
  \item The first step of mark and sweep algorithm, is to find the, so called,
    \textit{root nodes}. These are the nodes that are contained either in the
    stack, dump or globals parts of state. Function \texttt{findRootNodes} is
    designed for exactly this purpose. It scans stack, globals as well as dump
    and collects encountered nodes.
  \item In \textit{mark} phase, the entire heap, starting from roots is
    analysed, by means of recursive invocations of \texttt{markFrom} function,
    which wraps visited nodes with \texttt{NMarked} constructor.
  \item Finally, all memory is scanned from start to finish, examining all free
    or used blocks. Those without \texttt{NMarked} flag are not reachable
    by any program or data, and their memory is reclaimed. For nodes which are
    marked in-use, the flag is cleared again, preparing for the next
    cycle.
\end{itemize}


\chapter{Further work}

\bibliographystyle{alpha}
\bibliography{kivi}

\end{document}
